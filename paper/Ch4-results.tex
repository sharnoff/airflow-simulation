% Ch4-results.tex
%
% vim: set ft=tex:
\section{Results} \label{sec:results}

\todo{
    This section doesn't have anything yet. Here's an outline of what I'm currently planning on
    putting in it, in a rough order:
}

\begin{itemize}
    \item Numerical stability of extended experiments (i.e. ``we simulated 100/1000/10k seconds
        worth, and it stayed the same'')
        \begin{itemize}
            \item Currently, this would be extra-necessary to discuss because we aren't
                \textit{necessarily} starting in a solution to the system.
            \item Could also be good to have some comparisons of different starting volumes,
                just so that it's evident that they all converge very quickly.
        \end{itemize}
    \item Comparison of interpolation functions
        \begin{itemize}
            \item Good to get sooner so that it contextualizes the remaining experiments
            \item Also should mention here: reasoning for removing the kink from tanh,
                specifically the way that instantaneous changes in the state of the lungs causes
                instant reactions in the flow.
        \end{itemize}
    \item Stable flow characteristics of lungs with varying degradation
        \begin{itemize}
            \item Basically just want to point out the ways that the curve starts to become
                lopsided and out of phase from the pleural pressure, with particular emphasis on
                the way that the effect starts rapidly heightening as the constriction
                approaches 10\% (about~--~I haven't tested beyond that, which I should).
        \end{itemize}
    \item Onset and recovery from airflow constriction. Key things are:
        \begin{itemize}
            \item Just showing off the simple graphs, to get a sense of what's going on
            \item Slow recovery reduces strain~--~we can get abnormally high airflow if pleural
                pressure lines up with suddenly-reduced constriction 
            \item Relating to above: I'd like to have a version of the crazy graph~--~the one
                where it's all just ``degrade and recover'', with each startpoint being offset
                by 0.5s from the previous one. It would be really good to align that with some
                kind of chart showing the minimum and maximum flow velocity reached (on top of
                the normal min/max), to show the kind of chance of risk we run~--~and how much
                that reduces as the recovery becomes slower.
            \item Flow characteristics align with those from the current level of constriction
                (i.e. on the way to 10\%, the area around 20\% constriction tends to mimic the
                flow characteristics from 20\% at that point)
            \item The above points are probably three distinct sub-sections.
        \end{itemize}
\end{itemize}
