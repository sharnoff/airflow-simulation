% Appendices.tex
%
% vim: set ft=tex:

% helper derived from https://tex.stackexchange.com/a/252667
\titleformat{\section}{\normalfont\large\bfseries}{Appendix \thesection:}{1em}{}
\section{Extended equations} \label{appendix:equations}

\def\plmin{P_{\text{pl}_{\text{min}}}}
\def\plmax{P_{\text{pl}_{\text{max}}}}
\def\plinit{P_{\text{pl}_{\text{init}}}}

In \autoref{sec:simultaneous-equations}, the function controlling pleural pressure is referenced. It
is defined here, using a minimum $\plmin$, maximum $\plmax$ and initial pressure $\plinit$. The
period is $T$.

\begin{equation}
    P_{\text{pl}}(t_n) = M - A \cos\left(\frac{2\pi}{T} t_n + \cos^{-1}\left( \frac{M - \plinit}{A} \right)\right)
\end{equation}

\noindent where

\begin{align*}
    A &= \frac{\plmax - \plmin}{2}\hspace{1cm}&\text{(amplitude)} \\
    M &= \frac{\plmax + \plmin}{2}\hspace{1cm}&\text{(mean)}
\end{align*}

\noindent This function has the property that~--~if the initial value is not the minimum or maximum
value~--~it is always increasing at $t_n = 0$. If the initial value is the minimum,
$P_{\text{pl}}(t_n)$ is not technically increasing at $t_n = 0$, but it will be at $t_n = \epsilon$.
