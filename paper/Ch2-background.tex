% Ch2-background.tex
%
% vim: set ft=tex:
\section{Background}

\subsection{Physiology of the lungs}

Basic points:
\begin{itemize}
    \item air comes in through mouth, meets the lungs at the larynx
    \item lungs start off with the trachea, turns into bronchi/bronchial tubes
    \item eventually turns into bronchioles, alveoli \& capillary network (although we don't care
        \textit{as} much about these)
\end{itemize}

Want to talk about typical dimensions of each part here, as well as typical airflow \& how that's
observed (e.g. "what tests are used?") -- talk about intert-gas washout, among others.

\subsection{Clinical methods}

\begin{itemize}
    \item fMRI -- typically very expensive \& fairly low granularity: Good at getting an overview of
        the lung morphology, but resolution is typically low. (Not 100\% sure how accurate this
        assessment is, but I think I saw something about it.)
    \item Inert gas washout
\end{itemize}

The above are good for \textit{observing} changes in a patient, but have a couple shortcomings if we
want to analyze the effects of lung morphology. Firstly, observational tests cannot be performed
quickly; it would be prohibitively expensive to acquire the data for any kind of large-scale
analysis. Secondly, current clinical methods do not allow forced changes to patient lung morphology
(even without the risks and ethical considerations).

For those reasons, it is natural to turn to simulations -- in particular, computational models -- in
order to gain insight into impact on physiology and overall function from isolated changes within
the lungs.

\subsection{Prior computational models}

Talk briefly about proper fluid dynamics being too computationally expensive, use that to tie into
the model from Foy.

There's definitely other models, still need to look into those.
