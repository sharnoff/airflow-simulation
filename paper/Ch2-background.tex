% Ch2-background.tex
%
% vim: set ft=tex:
\section{Background}

\subsection{Physiology of the lungs}

At a high level, the physiology of the lungs can be divided into the few most significant
structures. This section does not discuss pulmonary blood circulation, or the mechanics of gas
exchange~--~this paper is primarily concerned with the flow of air in and out of the lungs.

The flow of air into the lungs begins with the diaphragm, a muscle below the lungs that contracts to
increase the volume of the \textit{thoracic cavity} (where lungs are housed). This increase in
volume causes the negative pressure that drives inspiration. Correspondingly, typical expiration is
driven by relaxation of the diaphragm and the associated increase in pressure in the thoracic
cavity.

Air flows into the body through the nose and mouth, meeting the trachea at the larynx, in the neck.
The trachea splits into the left and right bronchi~--~connecting to the left and right lungs
respectively. The bronchi split into a binary tree of progressively smaller bronchial tubes (the
\textit{tracheobronchial tree}), with the \textit{generation} of a bronchial tube referring to the
number of branches between it and the larynx. The smallest bronchial tubes terminate in a small
number of acini, small clusters of alveoli. Alveoli are the small, spherical air sacs that act as
the sites of gas exchange with the blood. A core technique used in this paper is the approximation
of one or many acini as spherical air sacs themselves, further described in
\autoref{sec:approximating-lungs}. Typical dimensions for all of the structures above are given in
\autoref{tab:lung-sizes}.

% FIXME: ^ need sources here
% https://en.wikipedia.org/wiki/Lung#/media/File:Secondary-pulmonary-lobule-illustration.jpg
%  ^ good diagram
% https://pubmed.ncbi.nlm.nih.gov/472520/
%  ^ "A statistical description of the human tracheobronchial tree geometry"

Beyond the structure of the lungs, there are also a number of measures of volume. \textit{Tidal
volume} (TV) refers to the volume of air moved in or out of the lungs during a typical breath, and
\textit{functional residual capacity} (FRC) refers to the total volume of air remaining in the lungs
after a normal expiration.\footnotemark\ Typical values for TV and FRC are also given in
\autoref{tab:lung-sizes}.

% FIXME: ^ source = https://www.ncbi.nlm.nih.gov/books/NBK545177/
%   ^ useful: https://pubmed.ncbi.nlm.nih.gov/12773331/
%   ^         https://www.ncbi.nlm.nih.gov/pmc/articles/PMC5980468/

\footnotetext{
    \textbf{N.B.:} TV and FRC both describe typical breaths; there are analogous terms for maximum
    capabilities (\textit{vital capacity} (VC) and \textit{residual volume} (RV)). Also notable is
    \textit{total lung capacity} (TLC; equal to VC + RV). These are included here for context, but
    they are not necessary for understanding the experiments in this paper.
}

\begin{table}[h]
\centering
    \begin{tabular}{ |c|c|c| }
    \hline
        Metric & Mean value (adult female) & Mean value (adult male) \\
        \hline \hline
        Trachea length & todo & todo \\ % FIXME: fill in the chart lol
        Trachea radius & todo & todo \\
        Bronchus length & todo & todo \\
        Bronchus radius & todo & todo \\
        Acinus radius & todo & todo \\
        Alveolus radius & todo & todo \\
        \hline
        TV & todo & todo \\
        FRC & todo & todo \\
        \hline
\end{tabular}
\caption{
    Typical sizes of various lung structures and volumes. Larger structures and effects tend to be
    more different in proportion between average males and females.
}
\label{tab:lung-sizes}
\end{table}

\subsection{Clinical methods}

There are a number of relevant clinical methods for measuring lung function, many of which will be
discussed in this section. Despite the clinical utiltiy however, there are certain limitations to
these methods that make them less well-suited to research, namely: difficulty with establishing
causation and cost per datapoint (either monetary, temporal, or both).

Reservations aside, current clinical tools for measuring lung function can essentially be grouped
into three categories: exhalation measurement: spirometry and inert-gas washout; oscillometry: FOT
and IOS; or imaging techniques: CT, PET, and MRI.

\textit{Spirometry} measures the volume and flow from a patient's maximal exhalation (after maximal
inhilation), producing volume-flow and volume-time curves. It is simple to perform, but the reliance
on maximal exhalation gives it a particularly low sensitivity. \textit{Inert-gas washout} instead
floods the lungs with an inert gas (e.g.~\ce{SF_6} or \ce{^3He}) before continuously measuring the
concentration of the gas exhaled through normal breathing. Measurements are either made over many
breaths (\textit{multiple-breath washout} (MBW)) or just one (\textit{single-breath washout} (SBW)).
Both MBW and SBW have a number of indices typically produced from the data, which correlate with
many lung diseases.

Both \textit{forced oscillation technique} (FOT) and \textit{impulse oscillometry} (IOS) apply
oscillations at the mouth and measure the resulting airflow and pressure. FOT uses controlled
pulses whereas IOS uses pseudo-random noise. From this, estimates of resistance and inertance of the
lungs are made, which have correlations to diseases such as asthma and COPD.

Finally, we have the imaging techniques: \textit{Magnetic resonance imaging} (MRI), \textit{computed
tomography} (CT), and \textit{positron emission tomography} (PET). Beyond immediately-visible
ailments (e.g. foreign objects or fluid buildup), these methods can also be used to estimate the
ventilation of air in each voxel of the image~--~providing metrics that can quantify overall lung
function (and highlight specific regions where function is degraded). However, these methods are
typically expenive (MRI) or dangerous in large amounts (CT and PET) and the timeline of resolution
improvements means that high-resolution imaging has not been available for as long as other methods.
The costs associated with these methods and the relative recency of high-resolution versions has
meant that there is also a relative gap in the literature linking imaging results to respiratory
disease classification.

This final point is one of the key reasons why computational models are so useful; high costs or low
availability hinder new research, and computer simulation can provide a simpler, cheaper method for
testing ideas. It is difficult to obtain large amounts of data for analysis when the underlying
methods are expensive or time-consuming, relative to the amount of data produced.\footnotemark

\footnotetext{
    To be clear here: methods like spirometry are relatively cheap and not overly time-consuming,
    but the amount of data each test generates is small; establishing complex relationships may
    require large amounts of data, regardless of the type of test used.
}

Also of note is that all of the above techniques are strictly observation with respect to the
condition of the lungs. Analyzing the effects of various morphological changes within the lungs is
difficult without the ability to directly effect those changes, but forced changes to patient lung
morphology are typically both risky and unethical.

For those reasons, it is natural to turn to simulations -- in particular, computational models -- in
order to gain insight into impact on physiology and overall function from isolated changes within
the lungs.

\subsection{Prior computational models}

This section is unfortunately brief~--~historically, computational models have been limited by the
available resources (they still are) and data to base them on. From the beginning, simulating
``full'' fluid dynamics with the Navier-Stokes equations has been both unnecessary and out of reach;
reasonable assumptions can be made about the flow of air within the lungs to simplify modelling
(see: \autoref{sec:approximating-lungs}), and models have increased in complexity over time to match
advancements in the speed of computers.

Early models represented the lungs with just a few elastic chambers, but advancements in
physiological data (particularly from Weibel, 1963) allowed later models to generate larger models
of the lungs, with the size now singularly limited by computational capabilities.

More recently, models have been partially constructed with results from patient imaging (up to
generations 6-10) with later generations generated by the parameters from Webiel (1963).

\todo{ the ``Weibel, 1963'' above needs to actually be a reference }
