% Ch2-background.tex
%
% vim: set ft=tex:
\section{Background}

\subsection{Physiology of the lungs}

At a high level, the physiology of the lungs can be divided into the few most significant
structures. This paper is primarily concerned with modelling airflow, so the intricacies of
pulmonary blood circulation or gas exchange are left for the reader to discover elsewhere.

\breakpars

The flow of air into the lungs begins with the diaphragm, a muscle below the lungs that contracts to
increase the volume of the \textit{thoracic cavity} (where lungs are housed). The inside of the
thoracic cavity and both lungs are lined with a thin membrane~--~creating a cavity around the lungs
when these two layers of membranes are separated. This cavity is the \textit{pleural cavity}, and
contains a small amount of \textit{pleural fluid}. The pressure of the pleural
fluid~--~\textit{pleural pressure}~--~directly drives the expansion and contraction of the lungs,
and its values are typically reported as relative to atmospheric pressure.

To keep the lungs inflated, pleural pressure is typically slightly less than atmospheric pressure;
the lungs can deflate if the pleural pressure is equal or greater to the air pressure inside the
lungs. As the diaphgram contracts to increase the volume of the thoracic cavity, the pressure in the
pleural cavity decreases, causing a greater negative pressure gradient, driving inspiration.
Correspondingly, typical expiration is driven by relaxation of the diaphragm and the associated
increase in pleural pressure from a smaller thoracic cavity.

Air flows into the body through the nose and mouth, meeting the trachea at the larynx, in the neck.
The trachea splits into the left and right bronchi~--~connecting to the left and right lungs
respectively. The bronchi split into a binary tree of progressively smaller bronchial tubes (the
\textit{tracheobronchial tree}), with the \textit{generation} of a bronchial tube referring to the
number of branches between it and the larynx. As the bronchial tubes become smaller, they switch
from conducting bronchioles (those that only transport air) to respiratory bronchioles, which attach
to and eventually terminate in a small small number of acini~--~small clusters of alveoli. The
first respiratory bronchioles are referred to as \textit{transitional bronchioles}, and are
generally found around generation 14.\cite{HaefeliBleuerWeibel1988}

Alveoli are the small, spherical air sacs that act as the sites of gas exchange with the blood. A
core technique used in this paper is the approximation of one or many acini as spherical air sacs
themselves, further described in \autoref{sec:approximating-lungs}. Typical dimensions for all of
the structures above are given in
\autoref{tab:lung-sizes}.

As individual alveoli fill with air, they expand in size. Alveoli are elastic, meaning expanding in
volume requires a greater pressure gradient between the air inside and the material outside. One way
of referring to this elasticity is by the \textit{compliance} of an alveolus~--~i.e. how ``willing''
it is to expand as the pressure gradient increases. There is also a parallel notion of ``total lung
compliance'', which describes the sum of the compliance of all alveoli in the lung, giving a measure
of how the entire lung \textbf{could} expand and contract in response to changing pleural
pressures.\footnotemark

\footnotetext{
    Technically speaking, the total lung compliance is related to what the volume of the lung would
    be if it were filled at atmospheric pressure, given a certain pleural pressure. In practice,
    there are other factors that prevent the system from equalizing in this way (e.g., airway
    resistance), and pleural pressure is not typically constant during normal breathing.
}

\begin{figure}[t]
    \centering
    \includegraphics[width=.8\textwidth,keepaspectratio]{figs/HumanRespiratorySystem.png}
    \caption{
        Diagram of the human respiratory system, at three levels of detail: high-level view of the
        whole system (A), zoomed into a pair of acini (B), and focusing on the site of gas exchange
        (C). Image sourced from Wikimedia Commons \\
        (\url{https://commons.wikimedia.org/wiki/File:Human\_respiratory\_system-NIH.PNG})
    }
\end{figure}

Beyond the structure of the lungs, there are also a number of measures of volume. \textit{Tidal
volume} (TV) refers to the volume of air moved in or out of the lungs during a typical breath, and
\textit{functional residual capacity} (FRC) refers to the total volume of air remaining in the lungs
after a normal expiration.\footnotemark\ Typical values for TV and FRC are also given in
\autoref{tab:lung-sizes}.

\footnotetext{
    \textbf{N.B.:} TV and FRC both describe typical breaths; there are analogous terms for maximum
    capabilities (\textit{vital capacity} (VC) and \textit{residual volume} (RV)). Also notable is
    \textit{total lung capacity} (TLC; equal to VC + RV). These are included here for context, but
    they are not necessary for understanding the experiments in this paper.
}

\begin{table}[ht]
    \caption{Average sizes and quantities of lung structures and volumes}
    \label{tab:lung-sizes}
    \centering
    \renewcommand{\arraystretch}{1.3}
    \begin{tabular}{ |c|c|c| }
    \hline
        Metric & Mean value (adult female) & Mean value (adult male) \\
        \hline \hline
        Trachea length & 9.8 cm \cite{KamelEtAl2009} & 10.5 cm \cite{KamelEtAl2009} \\
        Trachea radius$^a$ & 0.91 cm$^b$ \cite{Hoffstein1986} & 0.98 cm$^b$ \cite{Hoffstein1986} \\
        \hline
        Trans. bronchiole length & \multicolumn{2}{|c|}{ 1.33 mm \cite{HaefeliBleuerWeibel1988} } \\
        Trans. bronchiole radius & \multicolumn{2}{|c|}{ 0.25 mm \cite{HaefeliBleuerWeibel1988} } \\
        Acinus volume     & \multicolumn{2}{|c|}{ approx. 187 mm$^3$ $^c$ \cite{HaefeliBleuerWeibel1988} } \\
        Number of acini   & \multicolumn{2}{|c|}{ 30,000 \cite{WeibelEtAl2005} } \\
        Alveolus volume   & \multicolumn{2}{|c|}{ 0.0042 mm$^3$ \cite{OchsEtAl2004} } \\
        Number of alveoli & \multicolumn{2}{|c|}{ approx. 480,000,000$^d$ \cite{OchsEtAl2004} } \\
        \hline
        TV & 400 mL \cite{StatpearlsTidalVolume} & 500 mL \cite{StatpearlsTidalVolume} \\
        FRC & 2.64 L$^e$ & 3.50 L$^e$ \\
        % original FRC data:
        %          20-29yr , 30-39yr , 40-49yr , 50-59yr , 60-69yr , 70-80yr | unweighted avg
        %  male:   9x 3.36 , 7x 3.45 , 7x 3.50 , 8x 3.00 , 9x 3.79 , 10x3.88 | 2.635
        %  female: 9x 2.38 , 8x 2.54 , 7x 2.49 , 8x 2.54 , 9x 2.87 , 9x 2.99 | 3.496
        \hline
    \end{tabular}
    \newline
    \renewcommand{\arraystretch}{1}
    \footnotesize{
        % cheat-y way to get the footnotes left-aligned -- just have a lil table here! :)
        \begin{tabular}{p{14.5cm}}
            $^a$ Measurements are highly dependent on lung volume at time of
            measurement.\cite{Hoffstein1986} These values are at 50\% of VC, typically closer to FRC
            than TLC. Other studies provide higher-accuracy measurements, but at TLC.%
            \cite{SheelEtAl2009} \\
        $^b$ Calculated from airway cross-sectional area data, assuming circular cross-sections,
            \textit{then} averaged. \\
        $^c$ Data on this is limited; the actual mean may be quite different. \\
        $^d$ Variability in number of alveoli is very high (stdev $\approx 180 \times 10^6$), is
            unrelated to lung size, and density of alveoli is the same between boys and girls, so it
            is unclear whether there are sex differences. \cite{OchsEtAl2004, Thurlbeck1982} \\
        $^e$ Adapted from \cite{NederEtAl1999}, via unweighted average across all age groups. \\
        \end{tabular}
    }
\end{table}

\subsection{Clinical methods} \label{sec:clinical-methods}

There are a number of relevant clinical methods for measuring lung function, many of which will be
discussed in this section. Despite the clinical utiltiy however, there are certain limitations to
these methods that make them less well-suited to research, namely: difficulty with establishing
causation and cost per datapoint (either monetary, temporal, or both).

Reservations aside, current clinical tools for measuring lung function can essentially be grouped
into three categories: exhalation measurement: spirometry and inert-gas washout; oscillometry: FOT
and IOS; or imaging techniques: CT, PET, and MRI.

\textit{Spirometry} measures the volume and flow from a patient's maximal exhalation (after maximal
inhilation), producing volume-flow and volume-time curves. It is simple to perform, but the reliance
on maximal exhalation gives it a particularly low sensitivity. \textit{Inert-gas washout} instead
floods the lungs with an inert gas (e.g.~\ce{SF_6} or \ce{^3He}) before continuously measuring the
concentration of the gas exhaled through normal breathing. Measurements are either made over many
breaths (\textit{multiple-breath washout} (MBW)) or just one (\textit{single-breath washout} (SBW)).
Both MBW and SBW have a number of indices typically produced from the data, which correlate with
many lung diseases.

Both \textit{forced oscillation technique} (FOT) and \textit{impulse oscillometry} (IOS) apply
oscillations at the mouth and measure the resulting airflow and pressure. FOT uses controlled
pulses whereas IOS uses pseudo-random noise. From this, estimates of resistance and inertance of the
lungs are made, which have correlations to diseases such as asthma and COPD.

Finally, we have the imaging techniques: \textit{Magnetic resonance imaging} (MRI), \textit{computed
tomography} (CT), and \textit{positron emission tomography} (PET). Beyond immediately visible
ailments (e.g. foreign objects or fluid buildup), these methods can also be used to estimate the
ventilation of air in each voxel of the image~--~providing metrics that can quantify overall lung
function (and highlight specific regions where function is degraded). However, these methods are
typically expenive (MRI) or dangerous in large amounts (CT and PET) and the timeline of resolution
improvements means that high-resolution imaging has not been available for as long as other methods.
The costs associated with these methods and the relative recency of high-resolution versions has
meant that there is also a relative gap in the literature linking imaging results to respiratory
disease classification.

This final point is one of the key reasons why computational models are so useful; high costs or low
availability hinder new research, and computer simulation can provide a simpler, cheaper method for
testing ideas. It is difficult to obtain large amounts of data for analysis when the underlying
methods are expensive or time-consuming, relative to the amount of data produced.\footnotemark

\footnotetext{
    To be clear here: methods like spirometry are relatively cheap and not overly time-consuming,
    but the amount of data each test generates is small; establishing complex relationships may
    require large amounts of data, regardless of the type of test used.
}

Also of note is that all of the above techniques are strictly observation with respect to the
condition of the lungs. Analyzing the effects of various morphological changes within the lungs is
difficult without the ability to directly effect those changes, but forced changes to patient lung
morphology are typically both risky and unethical.

For those reasons, it is natural to turn to simulations -- in particular, computational models -- in
order to gain insight into impact on physiology and overall function from isolated changes within
the lungs.

\subsection{Prior computational models} \label{sec:prior-models}

This section is unfortunately brief~--~historically, computational models have been limited by the
available resources (they still are) and data to base them on. From the beginning, simulating
``full'' fluid dynamics with the Navier-Stokes equations has been both unnecessary and out of reach;
reasonable assumptions can be made about the flow of air within the lungs to simplify modelling
(see: \autoref{sec:approximating-lungs}), and models have increased in complexity over time to match
advancements in the speed of computers.

Early models represented the lungs with just a few elastic chambers, but advancements in
physiological data (particularly from \cite{Weibel1963}) allowed later studies to generate larger
models of the lungs, with the size now singularly limited by computational capabilities. However,
general models remained untouched for many years after. A brief survey of the time between shows: a
model of individual alveolar ducts from 1974 (\cite{Paiva1974}), a couple models specific to
multi-breath washout from 1990 and 2001 (\cite{VerbanckEtAl1990} and \cite{TawhaiHunter2001}), and a
model of \ce{^3He} gas diffusion~--~among many others not mentioned here.

More recently however, there was a 1-dimensional asymmetric branching model from 2012
(\cite{HenryEtAl2012}) and a \textit{different} 1-dimensional from 2017 that this paper builds on
(\cite{FoyEtAl2017}), using partial construction of from patient imaging (up to generations
6-10)~--~both of which model the lungs as a whole, directly.

In summary: computational models of the lungs as a whole are still relatively new; in this paper we
investigate some areas of untapped potential for these recent computational models.
