% Ch3-methods.tex
\section{Methods}

This section is mostly some temporary filler at the moment, just serving as a place to jot down the
formulae that we're using inside the simulation \textit{right now}.

\subsection{Simultaneous Equations}

There are essentially four simultaneous equations that we use to evaluate the state of the system at
a given point in time. This section provides a summary and brief description of each, the first of
which is the following:

\begin{equation}
    P_{\text{parent}(i)} - P_i = R(i) Q_i
\end{equation}

\noindent
which specifies that the pressure differential between the distal end of branch $i$ and its parent
must equal the pressure from the resistance from the flow through this branch $i$. For the ``root''
branch, $P_{\text{parent}(i)}$ is the pressure at the trachea~--~typically atmospheric pressure.

The resistance term $R(i)$ is defined as following function, as given by Pedley et al. (1970):

\begin{equation*}
    R(i) = \frac{2 \mu L_i c}{\pi r_i^4} \left( \frac{4 \rho |Q_i|}{\mu \pi L_i} \right)^{\frac{1}{2}}
\end{equation*}

The parenthesized term corresponds to the Reynold's number of the flow, scaled by the ratio of the
diameter of the branch to its length $L_i$. $r_i$ is the radius of branch $i$, $\mu$ is the
viscosity of the air, and $c = 1.85$ is a correction constant.

The second equation ensures incompressibility; the flow through a bifurcation must equal the sum of
the flow through its children:

\begin{equation}
    Q_i = \sum Q_{\text{child}}
\end{equation}

\noindent
where each \textit{child} refers to any branch $c$ with $\text{parent}(c) = i$.

The third equation maintains that the volume of an acinar region changes with the flow into or out
of it for the given timestep:

\begin{equation}
    V_i^t = V_i^{t-1} + dt Q^t_i
\end{equation}

\noindent
where $dt$ is the timestep size, $t$ refers to the current timestep, and $V_i$ is the volume of the
acinar region of branch $i$.

The final equation defines the elastic force of each acinar region, relating the pressure it exerts
on its branch to the volume of the region itself and the pressure outside it:

\begin{equation}
    P_i = \frac{1}{C_i} V_i + P_{pl}(t)
\end{equation}

\noindent
where $P_{pl}(t)$ is the pleural pressure (i.e. the ``pressure'' from the diaphragm, outside the
acinar region) at the current time and $C_i$ is the \textit{compliance} of the acinar region of
branch $i$. The pleural pressure changes over time to mimic human breathing patterns~--~hence why it
is parameterised by $t$.


\subsection{Units}

It is worth making explict the units used for each value. After careful consideration, these were
considered to provide the best spread of values with magnitude close to one, at which the accuracy
of floating-point accuracy is maximized. These are:

{
    % Add a bit more spacing within the rows of the table
    \renewcommand{\arraystretch}{1.6}
    \begin{table}[h]
    \centering
    \begin{tabular}{ |c|c| }
        \hline
        Type of thing ??? & Units \\
        \hline \hline
        Distance & $\text{m}$ \\
        \hline
        Volume & $\text{m}^3$ \\
        \hline
        Flow velocity & $\frac{ \text{m}^3 }{ \text{s} }$ \\
        \hline
        Density & $\frac{ \text{kg} }{ \text{m}^3 }$ \\
        \hline
        Pressure & Pascals $\left( \frac{ \text{kg} }{ \text{m} \cdot \text{s}^2 } \right)$ \\
        \hline
        Compliance & $\frac{ \text{m}^3 }{ \text{Pascal} }$ $\left( \frac{ \text{m}^4 \cdot \text{s}^2 }{ \text{kg} } \right)$ \\
        \hline
        Resistance & $\frac{ \text{kg} }{ \text{m}^4 \cdot \text{s} }$ \\
        \hline
        Viscosity & $\frac{ \text{kg} }{ \text{m} \cdot \text{s} }$ \\
        \hline
    \end{tabular}
    \end{table}
}
