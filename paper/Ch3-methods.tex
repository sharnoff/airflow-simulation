% Ch3-methods.tex
%
% vim: set ft=tex:
\section{Methods}

This section is mostly some temporary filler at the moment, just serving as a place to jot down the
formulae that we're using inside the simulation \textit{right now}.

\subsection{Simultaneous Equations}

There are essentially four simultaneous equations that we use to evaluate the state of the system at
a given point in time. This section provides a summary and brief description of each, the first of
which is the following:

\begin{equation}
    P_{\text{parent}(i)} - P_i = R(i) Q_i
\end{equation}

\noindent
which specifies that the pressure differential between the distal end of branch $i$ and its parent
must equal the pressure from the resistance from the flow through this branch $i$. For the ``root''
branch, $P_{\text{parent}(i)}$ is the pressure at the trachea~--~typically atmospheric pressure.

The resistance term $R(i)$ is defined as following function, as given by Pedley et al. (1970):

\begin{equation*}
    R(i) = \frac{2 \mu L_i c}{\pi r_i^4} \left( \frac{4 \rho |Q_i|}{\mu \pi L_i} \right)^{\frac{1}{2}}
\end{equation*}

The parenthesized term corresponds to the Reynold's number of the flow, scaled by the ratio of the
diameter of the branch to its length $L_i$. $r_i$ is the radius of branch $i$, $\mu$ is the
viscosity of the air, and $c = 1.85$ is a correction constant.

The second equation ensures incompressibility; the flow through a bifurcation must equal the sum of
the flow through its children:

\begin{equation}
    Q_i = \sum Q_{\text{child}}
\end{equation}

\noindent
where each \textit{child} refers to any branch $c$ with $\text{parent}(c) = i$.

The third equation maintains that the volume of an acinar region changes with the flow into or out
of it for the given timestep:

\begin{equation}
    V_i^t = V_i^{t-1} + dt Q_i^t
\end{equation}

\noindent
where $dt$ is the timestep size, $t$ refers to the current timestep, and $V_i$ is the volume of the
acinar region of branch $i$.

The final equation defines the elastic force of each acinar region, relating the pressure it exerts
on its branch to the volume of the region itself and the pressure outside it:

\begin{equation} \label{eq:acinar-pressure-volume-naive}
    P_i = \frac{1}{C_i} V_i + P_{pl}(t)
\end{equation}

\noindent
where $P_{pl}(t)$ is the pleural pressure (i.e. the ``pressure'' from the diaphragm, outside the
acinar region) at the current time and $C_i$ is the \textit{compliance} of the acinar region of
branch $i$. The pleural pressure changes over time to mimic human breathing patterns~--~hence why it
is parameterised by $t$.

\breakpars

At each timestep in the simulation, all of the simultaneous equations are grouped into a single
optimization function $f(\bm{x}) = (\text{EQ}_1..., \text{EQ}_2..., \text{EQ}_3..., \text{EQ}_4)$,
where each $\text{EQ}_i$ corresponds to the repeated instances of the $i$th equation above,
normalized so the right-hand-side equals zero. Thus the solution exists at $\bm{0}$, and we use
Euler's method to find an approximate solution, within $\norm{f(\bm{x})}^2 \le tol$ and
$\norm{dx}^2 \le tol$, with a tolerance of $10^{-6}$.

The input $\bm{x}$ is arranged with the values $(P..., Q..., V...)$ for each applicable
branch~--~i.e. using $P$ and $Q$ from all branches and $V$ from each acinar region.

\breakpars

The above descriptions are \textit{nearly} correct~--~they would work, but there were a few
adjustments made to the inputs and equations in practice for better numerical stability. These are
discussed in the \hyperref[sec:units-and-numerical-stability]{next section}.

\subsection{Units \& Numerical stability} \label{sec:units-and-numerical-stability}

It is worth making explict the units used for each value. After careful consideration, these were
considered to provide the best trade-off of familiar units and those with values of magnitude close
to one, where floating-point accuracy is maximized. As we will see momentarily, the spread was still
quite wide. The chosen units were:

{
    % Add a bit more spacing within the rows of the table
    \renewcommand{\arraystretch}{1.6}
    \begin{table}[h]
    \centering
    \begin{tabular}{ |c|c| }
        \hline
        Type of thing ??? & Units \\
        \hline \hline
        Distance & $\text{m}$ \\
        \hline
        Volume & $\text{m}^3$ \\
        \hline
        Flow velocity & $\frac{ \text{m}^3 }{ \text{s} }$ \\
        \hline
        Density & $\frac{ \text{kg} }{ \text{m}^3 }$ \\
        \hline
        Pressure & Pascals $\left( \frac{ \text{kg} }{ \text{m} \cdot \text{s}^2 } \right)$ \\
        \hline
        Compliance & $\frac{ \text{m}^3 }{ \text{Pascal} }$ $\left( \frac{ \text{m}^4 \cdot \text{s}^2 }{ \text{kg} } \right)$ \\
        \hline
        Resistance & $\frac{ \text{kg} }{ \text{m}^4 \cdot \text{s} }$ \\
        \hline
        Viscosity & $\frac{ \text{kg} }{ \text{m} \cdot \text{s} }$ \\
        \hline
    \end{tabular}
    \end{table}
}

One of the challenges with using these units is that some values are at a much greater magnitude
than the others. For example, the pressure inside each branch is close to atmospheric pressure~--~or
about $10^5$ Pascals, but pressure \textit{gradients} are typically much smaller.

In practice, this can mean that if the $dx$ from our Euler step is too small, the pressure won't
change; it doesn't have the necessary precision at that magnitude.

\breakpars

To mitigate this issue, we define two new values: $\hat{P}$ and $\hat{V}$, which are given by:

\begin{equation}
    \hat{P} = P - P_{\text{atm}}
\end{equation}

\noindent
where $P_{\text{atm}}$ is is atmospheric pressure; and:

\begin{align}
    \hat{V} & = V - V \vert_{P = P_{\text{atm}}} \\
            & = C (\hat{P} - P_{pl})
\end{align}

\noindent
Note that the definition of $\hat{V}$ would be the result of simply substituting $\hat{P}$ for $P$
in \ref{eq:acinar-pressure-volume-naive}. Applying these substitutions gives the following
equations, equivalent to their counterparts above:

\begin{equation}
    \hat{P}_{\text{parent}} - \hat{P}_i = R(i) Q_i
\end{equation}

\begin{equation}
    Q_i = \sum Q_{\text{child}}
\end{equation}

\begin{equation}
    \hat{V}_i^t = \hat{V}_i^{t-1} + dt Q_i^t
\end{equation}

\begin{equation}
    \hat{P}_i = \frac{1}{C_i} \hat{V}_i + P_{pl}(t)
\end{equation}

Representing the pressure and volume by their \textit{offset} from values at atmospheric pressure
causes them to cluster much closer to zero~--~the magnitude of the mean is significantly decreased,
relative to the variance of the values. This of course greatly improves the accuracy of each Euler
step.
