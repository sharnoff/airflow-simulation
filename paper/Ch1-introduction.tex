% Ch1-introduction.tex
%
% vim: set ft=tex:
\section{Introduction}

\subsection{Motivation} \label{sec:motivation}

Respiratory diseases account for more than 10\% of \textit{all} disability-adjusted life-years lost
due to any medical condition, second only to cardiovascular diseases (see: \cite{GlobalImpact}).
Because of this, any betterment of our understanding of the lungs and how they change from damage
can provide immediate benefits towards understanding one of the most significant categories of
disease.

In spite of this, there are relatively few existing methods for experimentation. Clinical
observations on live patients are necessarily limited, and common techniques~--~spirometry,
inert-gas washout, and MRI imaging~--~all have severe limitations that render them infeasible or
impossible to use for obtaining detailed data on the lungs at scale. And on top of that, the
difficulty of drawing inference from these methods is enhanced by the fact that they are purely
observational; in this paper, we are concerned with the effects of certain changes in lung
morphology (such as: tightening of the airways, stiffness in the expansion and contraction,
etc)~--~effects common to some of the most prevalent lung diseases, like asthma, cystic fibrosis,
and chronic obstructive pulmonary disease.

Of course, it would be unethical to \textit{induce} these changes in patients. However,
sufficiently-accurate computational models present a natural solution. By designing models that can
easily be arbitrarily deformed or otherwise altered, we create the opportunity to efficiently
investigate how targeted changes in lung morphology affect both lung functioning as a whole and the
stresses placed on individual regions.

Historically, computational models have been severely limited by available resources, but recent
advancements in modelling the lungs have shown new methods to be both efficient and accurate,
opening the door to a wide range of possibilities.

\subsection{Contributions to the field} \label{sec:contributions}

This paper introduces a new tool for generating computational lung models, simulating them, and
changing their morphology mid-simulation. To demonstrate the utility of this tool, we investigate
the response of the lungs to various morphological changes, providing this data over a smooth range
of possible changes~--~allowing trends to be plainly visible and inference to be easily made.
