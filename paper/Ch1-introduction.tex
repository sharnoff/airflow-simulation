% Ch1-introduction.tex
%
% vim: set ft=tex:
\section{Introduction}

\subsection{Motivation}

Respiratory diseases account for more than 10\% of \textit{all} disability-adjusted life-years lost
due to any medical condition, second only to cardiovascular diseases. ~\cite{GlobalImpact} Because
of this, any betterment of our understanding of the lungs and how they change from damage has
immediate benefits towards understanding one of the most significant categories of disease.

In spite of this, there are relatively few existing methods for experimentation. Clinical
observations on live patients are necessarily limited, and common techniques~-- spirometry,
inert-gas washout, and fMRI imaging~-- all have severe limitations that render them infeasible or
impossible to use for obtaining detailed data on the lungs at scale. And on top of that, the
difficulty of drawing inference from these methods is enhanced by the fact that they are purely
observational; in this paper, we are concerned with the effects of certain changes in lung
morphology (such as: tightening of the airways, stiffness in the expansion and contraction, etc.).

Of course, it would be unethical to \textit{induce} these changes in patients. However,
sufficiently-accurate computational models present a natural solution. By designing models that can
easily be arbitrarily deformed or otherwise altered, we create the opportunity to efficiently
investigate how targeted changes in lung morphology affect both lung functioning as a whole and the
stresses placed on individual regions.

\todo{I'd like to add something along the lines of: ``historically, computational models have been
too expensive for experimentation without specialized equipment, but recent developments (i.e. Foy)
have show other methods for simulating airflow are both efficient and accurate.''}

\subsection{Contributions to the field}

This paper introduces a new tool for simulating and observing changes to the lungs, and their
precise effects.

\todo{Something about: ``this is useful to people looking to find new results about \textit{how} the
lungs get impacted by various diseases.'' Would like to also say: ``we have investigated the effects
of clinically-observed symptoms from a couple diseases, to showcase the utility of this tool''}
