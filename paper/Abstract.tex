% Abstract.tex
%
% vim: set ft=tex:

{\Large \textbf{Abstract}}

\vspace{0.5cm}

Detecting changes in human lung morphology and determining its effects on lung function requires
significant time committment per patient, so statistical analysis on many individuals is infeasible.
Computational models of the lung are therefore a natural choice for researching the effects of
altered lung morphology, with reference to existing lung function tests.

Accurate computational models also allow investigation into properties of the lungs that cannot
feasibly be measured; e.g., increased internal stress in one location from damaged airways
elsewhere.

This paper builds on recent advancements in modelling airflow in the lungs (\textlangle reference to
Foy\textrangle) to produce an efficient, accurate model of the lungs that supports simple
alterations to the simulated morphology.  We then use this model to determine the strains placed on
the rest of the lungs by various kinds of constricted or damaged airways.
