% Ch5-conclusions.tex
%
% vim: set ft=tex:
\section{Conclusions}

\subsection{Summary of key results} \label{sec:summary-of-key-results}

In this paper, we have used recent, established mathematical models to investigate the
characteristics of airflow in the lungs under a variety of conditions. In particular, we've
demonstrated a clear association between maximum acceleration in flow and airway constrict, in a
way that is not as visible in just the maximum speed of flow reached.

From there, we introduce the ``constrict and recover'' pattern, where the airways in an
initially-healthy lung are gradually constricted to a desired level and after a delay, they
gradually return back to the healthy state. The flow characteristics as the constriction level
changes appear \textit{similar} to a composition of the flows from intermediate constriction levels,
and we show the way that this effect is rooted in the basic mechanics of the system, explaining why
this causes the airflow to appear abnormal at certain points.

Extrapolating from this, we find a particular worst-case scenario wherein the quick correction from
one volume to the ``target'' volume for the pleural pressure cauess an extreme spike in the airflow
velocity.

And finally, we show by asymmetric constriction that the primary source of resistance is not coming
from the trachea. This is not \textit{necessarily} incorrect, but it is certainly interesting,
considering that prior studies have shown that the majority of airflow resistance in the lungs comes
from the larger conducting airways. One possible explanation for this is the exact method we were
using for whole lung constriction~--~in practice it \textit{may} be more likely that typical airway
constriction (e.g., from asthma) has a greater proportional effect on the larger airways.

\subsection{Limitations and further work} \label{sec:limitations-and-further-work}

There are of course a number of areas of possible improvement for the our experiments~--~a number of
which have been briefly mentioned already, but they bear repeating. From a purely technical
standpoint, it's certainly possible to obtain more efficient models with a better matrix
factorisation algorithm; GMRES will likely perform much better, but perhaps the most significant
gains in simulation size and speed would come from parallel factorisation.

There are also a number of avenues for improvement in the accuracy of the simulation: An immediate
improvement would be to use models synthesized from lung scans, as in \cite{FoyEtAl2017}. But
particularly as we increase constriction, it is also unreasonable to expect that the pleural
pressure will remain the same~--~a model of pleural pressure that reacts to the lung volume in a
similar way to humans may provide a more accurate result when looking at severe constriction.

\breakpars

Looking forward, there are possible applications of this research outside of this field~--~fast and
efficient simulation provides a key opportunity to use this tool to provide data for data-intensive
applications, like machine learning (in \cite{SuoEtAl2021}, for example, data was gathered from a
physical system where it could have been done by modifying this software).

There are also possibilities within this space. Given the results on maximum flow acceleration,
there may room for new multiple-breath washout indices that are more sensitive to constriction that
is not as severe.
